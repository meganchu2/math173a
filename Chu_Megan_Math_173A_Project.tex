\documentclass[12pt]{article}
\usepackage{amsmath}
\usepackage{amsfonts}
\usepackage{amssymb}
\usepackage{amsthm}
\usepackage{graphicx}
\usepackage{hyperref}
\usepackage[latin1]{inputenc}
\usepackage{enumerate}
\usepackage{color}

\title{MATH 173A Project}
\author{Megan Chu}
\date{07/27/18}

\documentclass{article} \usepackage[left=2.54cm, right=2.54cm, top=2.54cm]{geometry}
\begin{document}
\maketitle
\centerline{Problem Solving (Latex) 3.10 and 3.19}\pagebreak

% Ex.3.10
\noindent Ex.3.10. A decryption exponent for an RSA public key $(N, e)$ is an integer $d$ with the property that $a^{de} \equiv a (mod N)$ for all integers $a$ that are relatively prime to $N$.

\begin{enumerate}[a)]




  % Ex.3.10.(a)
  \item Suppose that Eve has a magic box that creates decryption exponents for $(N, e)$ for a fixed modulus $N$ and for a large number of different encryption exponents $e$. Explain how Eve can use her magic box to try to factor $N$.
  
  \color{blue}
  \item[\underline{\textit{Solution:}}] Eve knows a large number of different encryption exponents, which we will denote as $e_{1}, e_{2}, ..., e_{k}$.  Using the magic box, Eve can obtain all decryption exponents for the set of encryption exponents, which we will denote as $d_{1}, d_{2}, ..., d_{k}$, and where $d_n$ is the decryption exponent for the RSA public key $(N, e_{n})$ where $n \in \{1, 2, ..., k\}$.  \\
  
  In general, $N$ can be factored into primes $p_{1}p_{2}...p_{r}$, such that $N=p_{1}p_{2}...p_{r}$.  Eve also knows that 
  $$a^{de} \equiv a (\text{mod } N) \equiv a (\text{mod } p_{1}) \equiv a (\text{mod } p_{2}) \equiv ... \equiv a (\text{mod } p_{r}),$$
  since $N$ is a multiple of any one of the primes in the set $p_{1}, p_{2}, ..., p_{r}$.  \\
  
  Then, $de \equiv 1 (\text{mod } (p_{1}-1)) \equiv 1 (\text{mod } (p_{2}-1) \equiv ... \equiv 1 (\text{mod } (p_{r}-1))$ must be true, so that the following statements can also hold true. \\
  $$a^{de} \equiv a^{k(p_{1}-1)+1} \equiv (a^{p_{1}-1})^{k}a^{1} \equiv 1^{k}a \equiv a (\text{mod } p_{1}),$$
  $$a^{de} \equiv a^{k(p_{2}-1)+1} \equiv (a^{p_{2}-1})^{k}a^{1} \equiv 1^{k}a \equiv a (\text{mod } p_{2}),$$ 
  $$...,$$
  $$a^{de} \equiv a^{k(p_{r}-1)+1} \equiv (a^{p_{r}-1})^{k}a^{1} \equiv 1^{k}a \equiv a (\text{mod } p_{r}).$$
  \begin{align*}
  \text{This also implies that }&de \equiv 1 (\text{mod } (p_{1}-1)(p_{2}-1)...(p_{r}-1));\phantom{abcdefghijklmnopqrstuvwxy} \\
  \text{equivalently, }&de-1 \equiv 0 (\text{mod } \phi (N)); \\
  \text{equivalently, }&\phi (N) | de-1.  
  \end{align*}
  
  Since we know that the RSA public key cryptosystem involves $N$ as a product of 2 primes, we can say $N=pq$ for primes $p$ and $q$.  Eve can now compute the following value: $b=gcd(d_{1}e_{1}-1, d_{2}e_{2}-1, ..., d_{k}e_{k}-1)$, with confidence that $b$ is a factor of $\phi (N)=(p-1)(q-1)$. Note that we assume we have enough encryption/decryption exponent pairs such that $b$ will be a small value less than $(p-1)(q-1)$.\\ 
  
  Remark 3.11 from the textbook tells us we can expand $\phi (N)$ like so:
  \begin{align*}
  \phi (N)&=(p-1)(q-1)\\
  &=pq-p-q+1\\
  &=N-(p+q)+1.  \\
  \end{align*}
  
  Since Eve knows the value of $N$, if she can find out the value of $p+q$, she can solve for the value $(p-1)(q-1)$.  Furthermore, if Eve knows the values of $p+q$ and $N=pq$, she can use the quadratic formula to solve the equation 
  \begin{equation} \tag{3.10.1}
  x^{2}-(p+q)x+pq=(x-p)(x-q)=0
  \end{equation}
  for its roots $p$ and $q$.\\
  
  Eve can try random values of $p+q$ to see if they result in (3.10.1) having integer factors.  This method can be further simplified since Eve knows $b$ is a factor of $(p-1)(q-1)$.  In other words, $bi=(p-1)(q-1)$ for some $i \in \mathbb{Z}$.  We also see that 
  \begin{align*}
  (p-1)(q-1)=N-(p+q)+1 \\
  &\Leftrightarrow (p+q)=N+1-(p-1)(q-1) \\
  &\Leftrightarrow (p+q)=N+1-bi \text{ for some } i \in \mathbb{Z}.
  \end{align*}
  Thus, Eve can try increasing integers for i starting from 1, solve for $(p+q)$, and try to factor (3.10.1).  Furthermore, Eve can estimate i since 
  \begin{align*}
  bi=&(p-1)(q-1) \\
  &\Rightarrow bi < (p)(q) \\
  &\Rightarrow bi < N \\
  &\Rightarrow i < \frac{N}{b}.
  \end{align*}
  Thus, Eve only has to test decreasing values of $i$ starting from the first integer less than $\frac{N}{b}$ and stopping at $i=1$.  Once she finds a value for $i$ giving a $(p+q)$ that allows for factorization of (3.10.1) into integer roots, she can use the quadratic formula to find $p$ and $q$; hence, factoring $N$.
  \color{black}
  \pagebreak
  
  
  
  
  % Ex.3.10.(b)
  \item Let $N$ = 38749709. Eve's magic box tells her that the encryption exponent $e$ = 10988423 has decryption exponent $d$ = 16784693 and that the encryption exponent $e$ = 25910155 has decryption exponent $d$ = 11514115. Use this information to factor $N$.
  
  \color{blue}
  \item[\underline{\textit{Solution:}}] Following (a), we calculate $b$ and $i$
  \begin{align*}
  b=&gcd(d_{1}e_{1}-1, d_{2}e_{2}-1)\\
  &=gcd(16784693\cdot10988423-1, 11514115\cdot25910155-1)\\
  &=gcd(184437306609138, 298332504337824)\\
  &=19368558\text{, using the wolframalpha gcd calculator}
  \end{align*}
  $$i < \frac{N}{b} < \frac{38749709}{19368558} < 2.00065017747 \Rightarrow i \approx 2$$
  
  Thus, we try i = 2:
  $$p+q=N+1-b \cdot i=38749709+1-19368558\cdot2=12594$$
  Then, $a'=1,b'=12594,c'=N=38749709$, and $x^{2}-12594x+38749709=0$ has solutions \\
  $$x=\frac{b'\pm\sqrt{b'^{2}-4a'c'}}{2a'}=\frac{12594\pm\sqrt{3610000}}{2}= 5347 \text{ or } 7247$$\\
  
  Thus, \fbox{$N=5347\cdot7247$}.
  \color{black}
  \pagebreak




  % Ex.3.10.(c)
  \item Let $N$ = 225022969. Eve's magic box tells her the following three encryption/decryption pairs for $N$:\\
  (70583995, 4911157),\hfill(173111957, 7346999),\hfill(180311381, 29597249).\\
  Use this information to factor $N$.
  
  \color{blue}
  \item[\underline{\textit{Solution:}}] We calculate $b$ and $i$
  \begin{align*}
  b=&gcd(d_{1}e_{1}-1, d_{2}e_{2}-1, d_{3}e_{3}-1)\\
  &=gcd(4911157\cdot70583995-1, 7346999\cdot173111957-1, 29597249\cdot180311381-1)\\
  &=gcd(346649081132214, 1271853374967042, 5336720840990868)\\
  &=37498566\text{, using the wolframalpha gcd calculator}
  \end{align*}
   $$i < \frac{N}{b} < \frac{225022969}{37498566} < 6.00084197886 \Rightarrow i \approx 6$$\\
  
  To avoid trying many values of i by hand, we use a computer program to do such calculations and we see that i = 6 indeed gives integer solutions.  When i = 6, \\
  $$p+q=N+1-b \cdot i=225022969+1-37498566\cdot6=31574$$\\
  Then, $a'=1,b'=31574,c'=N=225022969$, and $x^{2}-31574x+225022969=0$ has solutions\\
  $$x=\frac{b'\pm\sqrt{b'^{2}-4a'c'}}{2a'}=\frac{31574\pm\sqrt{96825600}}{2}= 10867 \text{ or } 20707$$\\
  
  Thus, \fbox{$N=10867\cdot20707$}.
  \color{black}
  \pagebreak




  % Ex.3.10.(d)
  \item Let $N$ = 1291233941. Eve's magic box tells her the following three encryption/decryption pairs for $N$:\\
  (1103927639, 76923209),\hfill(1022313977, 106791263),\hfill(387632407, 7764043).
  Use this information to factor $N$.
  
  \color{blue}
  \item[\underline{\textit{Solution:}}] We calculate $b$ and $i$
  \begin{align*}
  b=&gcd(d_{1}e_{1}-1, d_{2}e_{2}-1, d_{3}e_{3}-1)\\
  &=gcd(76923209\cdot1103927639-1, 106791263\cdot1022313977-1, 7764043\cdot387632407-1)\\
  &=gcd(84917656495673550, 109174200786382950, 3009594676141500)\\
  &=129112350\textrm{, using the wolframalpha gcd calculator}
  \end{align*}
  $$i < \frac{N}{b} < \frac{1291233941}{129112350} < 10.0008553868 \Rightarrow i \approx 10$$
  
  To avoid trying many values of i by hand, we use a computer program to do such calculations and we see that i = 10 indeed gives integer solutions.  When i = 10, \\
  $$p+q=N+1-b\cdot i=1291233941+1-129112350\cdot10=110442$$\\
  Then, $a'=1,b'=110442,c'=N=1291233941$, and $x^{2}-110442x+1291233941=0$ has solutions\\
  $$x=\frac{b'\pm\sqrt{b'^{2}-4a'c'}}{2a'}=\frac{110442\pm\sqrt{7032499600}}{2}= 13291 \text{ or } 97151$$\\
  
  Thus, \fbox{$N=13291\cdot97151$}.
  \color{black}
  
\end{enumerate}\pagebreak









% Ex.3.19
\noindent Ex.3.19. We noted in Sect. 3.4 that it really makes no sense to say that the number $n$ has probability 1/ln($n$) of being prime. Any particular number that you choose either will be prime or will not be prime; there are no numbers that are 35\% prime and 65\% composite! In this exercise you will prove a result that gives a more sensible meaning to the statement that a number has a certain probability of being prime. You may use the prime number theorem (Theorem 3.21) for this problem.
\begin{enumerate}[a)]




  % Ex.3.19.(a)
  \item Fix a (large) number $N$ and suppose that Bob chooses a random number $n$ in the interval $\frac{1}{2}N \leq n \leq \frac{3}{2}N$. If he repeats this process many times, prove that approximately 1/ln($N$) of his numbers will be prime. More precisely, define \\
  \begin{align*}
  P(N)&=\frac{\text{number of primes between} \frac{1}{2}N \text{ and } \frac{3}{2}N}{\text{number of integers between} \frac{1}{2}N \text{ and } \frac{3}{2}N} (inclusive)\\
  &=\left[\substack{\text{Probability that an integer } n \text{ in the} \\ \text{interval } \frac{1}{2}N \leq n \leq \frac{3}{2}N \text{ is a prime number}}\right],
  \end{align*}
  and prove that \\
  $$\lim\limits_{N \to \infty} \frac{P(N)}{\text{1/ln}(N)}=1.$$
  This shows that if $N$ is large, then $P(N)$ is approximately 1/ln($N$).
  
  \color{blue}
  \item[\underline{\textit{Solution:}}] Supposing that $N$ is divisible by 2, we see that the number of integers between $\frac{1}{2}N \text{ and } \frac{3}{2}N$ is equal to $\frac{3}{2}N-\frac{1}{2}N+1=\frac{2}{2}N+1=N+1$.  Even if $N$ is not divisible by 2, the number of integers between $\frac{1}{2}N \text{ and } \frac{3}{2}N$ is still at least $N-1$.  Regardless of the divisibility of $N$, since $N$ approaches $\infty$ in the limit we want to prove, and since the number of integers between $\frac{1}{2}N \text{ and } \frac{3}{2}N$ has the value of $N+$ ``some constant", we can simplify the number of integers between $\frac{1}{2}N \text{ and } \frac{3}{2}N$ to just \fbox{$N$} without affecting the final result of the limit.  \\
  
  Section 3.4.1 in the textbook says that by definition, \\
  $\pi(X)=(\text{\# of primes } p \text{ satisfying } 2 \leq p \leq X)$. Thus, 
  $$\pi(\frac{3}{2}N)=(\text{\# of primes between 2 and } \frac{3}{2}N)$$
  $$\pi(\frac{1}{2}N)=(\text{\# of primes between 2 and } \frac{1}{2}N)$$
  It follows that \\
  $$\text{\# of primes between } \frac{1}{2}N \text{ and } \frac{3}{2}N=\fbox{\pi(\frac{3}{2}N)-\pi(\frac{1}{2}N)}$$
  
  Thus, as $N \to \infty \text{, } P(N) \to \frac{\pi(\frac{3}{2}N)-\pi(\frac{1}{2}N)}{N}$, and we have, \\
  $$\lim\limits_{N \to \infty} \frac{P(N)}{\text{1/ln}(N)}=\lim\limits_{N \to \infty} \frac{\frac{\pi(\frac{3}{2}N)-\pi(\frac{1}{2}N)}{N}}{\text{1/ln}(N)}=\lim\limits_{N \to \infty} \frac{\pi(\frac{3}{2}N)-\pi(\frac{1}{2}N)}{N/\text{ln}(N)}$$
  
  Theorem 3.21 (The Prime Number Theorem) from the textbook tells us that \\
  $\lim\limits_{N \to \infty} \frac{\pi(X)}{X/\text{ln}(N)}=1$.  Hence, we can replace the terms $\pi(\frac{3}{2}N)$ and $\pi(\frac{1}{2}N)$ with $\frac{3}{2}N/\text{ln}(\frac{3}{2}N)$ and $\frac{1}{2}N/\text{ln}(\frac{1}{2}N)$, respectively, without affecting the final result of the limit. Thus, 
  \begin{align*}
  \lim\limits_{N \to \infty} \frac{\pi(\frac{3}{2}N)-\pi(\frac{1}{2}N)}{N/\text{ln}(N)
  &=\lim\limits_{N \to \infty} \frac{\frac{3}{2}N/\text{ln}(\frac{3}{2}N)-\frac{1}{2}N/\text{ln}(\frac{1}{2}N)}{N/\text{ln}(N)}\\
  &=\lim\limits_{N \to \infty} \frac{\frac{3}{2}/\text{ln}(\frac{3}{2}N)-\frac{1}{2}/\text{ln}(\frac{1}{2}N)}{1/\text{ln}(N)}}\\
  &=\lim\limits_{N \to \infty} \frac{\frac{3}{2}/\text{ln}(\frac{3}{2}N)}{1/\text{ln}(N)}-\frac{\frac{1}{2}/\text{ln}(\frac{1}{2}N)}{1/\text{ln}(N)}\\
  &=\lim\limits_{N \to \infty} \frac{\frac{3}{2}\text{ln}(N)}{\text{ln}(\frac{3}{2}N)}-\frac{\frac{1}{2}\text{ln}(N)}{\text{ln}(\frac{1}{2}N)}\\
  &=\lim\limits_{N \to \infty} \frac{3\text{ln}(N)}{2\text{ln}(\frac{3}{2}N)}-\frac{1\text{ln}(N)}{2\text{ln}(\frac{1}{2}N)}
  \end{align*}
  
  Clearly, as $N \to \infty$ the values of ln$(N)$, ln$(\frac{3}{2}N)$, and ln$(\frac{1}{2}N)$ plateau, since all three funtions are $O($ln$(N))$ and have logarithmic growth.  In other words, as $N \to \infty$, the values of ln$(N)$, ln$(\frac{3}{2}N)$, and ln$(\frac{1}{2}N)$ will converge/approach each other.  Thus, we can cancel out these terms without affecting the final result of the limit.
  \begin{align*}
  \lim\limits_{N \to \infty} \frac{3\text{ln}(N)}{2\text{ln}(\frac{3}{2}N)}-\frac{1\text{ln}(N)}{2\text{ln}(\frac{1}{2}N)}&=\lim\limits_{N \to \infty} \frac{3}{2}-\frac{1}{2}\\
  &=\frac{3}{2}-\frac{1}{2}\\
  &=\frac{2}{2}\\
  &=1
  \end{align*}
  We have shown $\lim\limits_{N \to \infty} \frac{P(N)}{\text{1/ln}(N)}=1$. \fbox{}
  \color{black}
  \pagebreak
  
  
  
  
  % Ex.3.19.(b)
  \item More generally, fix two numbers $c_{1}$ and $c_{2}$ satisfying $c_{2} > c_{1} > 0$. Bob chooses random numbers $n$ in the interval $c_{1}N \leq n \leq c_{2}N$. Keeping $c_{1}$ and $c_2$ fixed, let 
  $$P(c_{1},c_{2};N)=\left[\substack{\text{Probability that an integer } n \text{ in the inter-} \\ \text{val } c_{1}N \leq n \leq c_{2}N \text{ is a prime number}}\right].$$
  In the following formula, fill in the box with a simple function of N so that the statement is true: 
  $$\lim\limits_{N \to \infty} \frac{P(c_{1},c_{2};N)}{\framebox[1cm]{}}=1.$$
  
  \color{blue}
  \item[\underline{\textit{Solution:}}] From part (a), we proved that if $c_{2}=\frac{3}{2}$ and $c_{1}=\frac{1}{2}$, then, $\lim\limits_{N \to \infty} \frac{P(c_{1},c_{2};N)}{1/\text{ln}(N)}=\frac{3}{2}-\frac{1}{2}=c_{2}-c_{1}=1$.  Seeing that\\
  \begin{equation} \tag{3.19.1}
  \lim\limits_{N \to \infty} \frac{P(c_{1},c_{2};N)}{1/\text{ln}(N)}=c_{2}-c_{1}\\
  \end{equation}
  and using the knowledge that $\frac{c_{2}-c_{1}}{c_{2}-c_{1}}=1$, we divide both sides of (3.19.1) by $c_{2}-c_{1}$.\\
  \begin{align*}
  &\frac{1}{c_{2}-c_{1}}\lim\limits_{N \to \infty} \frac{P(c_{1},c_{2};N)}{1/\text{ln}(N)}=\frac{c_{2}-c_{1}}{c_{2}-c_{1}}=1\\
  \Leftrightarrow&\lim\limits_{N \to \infty} \left(\frac{1}{c_{2}-c_{1}}\right)\frac{P(c_{1},c_{2};N)}{1/\text{ln}(N)}=1\\
  \Leftrightarrow&\lim\limits_{N \to \infty} \frac{P(c_{1},c_{2};N)}{\fbox{(c_{2}-c_{1})/\text{ln}(N)}}=1 
  \end{align*}
  
  Thus, $\fbox{(c_{2}-c_{1})/\text{ln}(N)}$ correctly fills in the box so that $\lim\limits_{N \to \infty} \frac{P(c_{1},c_{2};N)}{\framebox[1cm]{}}=1$ is true.
  \color{black}
  
\end{enumerate}

\end{document}
